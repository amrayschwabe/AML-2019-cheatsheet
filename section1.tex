\section*{Probabilities}
\subsection*{Expectation}
$\mathbbm{E}[X]=\int_{\Omega}xf(x)\di x=\int_{\omega}x\mathbb{P}[X{=}x]\di x$ \\
$\mathbb{E}_{Y|X}[Y]=\mathbb{E}_{Y}[Y|X]$\\
$\mathbb{E}_{X,Y}[f(X,Y)]=\mathbb{E}_{X}\mathbb{E}_{Y|X}[f(X,Y)|X]$
% $\mathbb{E}_{Y|X}[f(X,Y)|X]{=}\int_\mathbb{R}f(X,y)\mathbb{P}(y|X)\di y$

\subsection*{Variance \& Covariance}
$\mathbb{V}(X){=}\mathbb{E}[(X{-}\mathbb{E}[X])^2]{=}\mathbb{E}[X^2]{-}\mathbb{E}[X]^2$\\
$\mathbb{V}[X+Y]{=}\mathrm{Var}[X]+\mathrm{Var}[Y] X,Y \,\text{iid} \quad
\mathbb{V}(AX) = A \mathbb{V}(X) A^T$ 
$\mathbb{V}[\alpha X]=\alpha^2\mathrm{Var}[X]$

$\mathrm{Cov}(X,Y)=\mathbb{E}[(X-\mathbb{E}[X])(Y-\mathbb{E}[Y])]$

\subsection*{Conditional Probabilities \& Bayes}
$\mathbb{P}[X|Y]=\frac{\mathbb{P}[X,Y]}{\mathbb{P}[Y]}=\frac{\mathbb{P}[Y|X]\mathbb{P}[X]}{\mathbb{P}[Y]}$
$\rightarrow$ can be found by taking $\mathbb{P}[X,Y]$ take Y const.

\subsection*{Distributions}
$\mathcal{N}(x|\mu, \sigma^2)=\frac{e^{-(x-\mu)^2/(2\sigma^2)}}{\sqrt{2\pi\sigma^2}}$\\
$\mathcal{N}(x|\bm{\mu}, \bm{\Sigma})= \frac{e^{-\frac{1}{2}(\mathbf{x}-\bm{\mu})^T\bm{\Sigma}^{-1}(\mathbf{x}-\bm{\mu})}}{(2\pi)^{D/2}|\bm{\Sigma}|^{1/2}} $\\
$\Sigma = \mathbb{E}[(x_i-\mu)(x_i-\mu)^T] = \mathbb{E}[x_ix_i^T]-\mu\mu^T \rightarrow \mathbb{E}[x_ix_i^T] = \Sigma + \mu \mu^T$ and for $i\neq j$ $\mathbb{E}[x_ix_j^T] = \mu\mu^T$
Also useful: $\frac{1}{N} \sum_{i=1}^N \mathbb{E}[x_i\hat\mu^T] = \frac{1}{N^2} \sum_{i=1}^N \sum_{j=1}^N \mathbb{E}[x_ix_j] $ $\mathrm{Ber}(x|\theta){=}\theta^x (1{-}\theta)^{(1-x)}$ \\
Sigmoid: $\sigma(x)=1/(1+e^{-x})$
\subsection*{Chebyshev \& Consistency}
$\mathbb{P}(|X-\mathbb{E}[X]|\geq \epsilon)\leq \frac{\mathbb{V}[X]}{\epsilon^2}$\\
\subsection*{Jensen Inequality}
$f(\sum_{i=1}^n \lambda_i x_i) \leq \sum_{i=1}^n \lambda_i f(x_i)$

\subsection*{Cramer Rao lower bound}
$\mathrm{Var}[\hat{\theta}]\geq \mathcal{I}_n(\theta)^{-1}$\\
$\mathcal{I}_n(\theta) = -\mathbb{E}[\frac{\partial^2 \mathrm{log}p[\mathcal{X}_n|\theta]}{\partial \theta^2}]$, $\hat{\theta}$ unbiased\\
Efficiency of $\hat{\theta}$: $e(\theta_n)=\frac{1}{\mathrm{Var}[\hat{\theta}_n]\mathcal{I}_n(\theta)}$\\
$e(\theta_n) = 1$ (efficient)\\
$lim_{n\rightarrow\infty}e(\theta_n) = 1$ (asymp. efficient)

\subsection*{Derivatives}
$\frac{\partial}{\partial \mathbf{x}}(\mathbf{b}^\top \mathbf{x}) = \frac{\partial}{\partial \mathbf{x}}(\mathbf{x}^\top \mathbf{b}) = \mathbf{b}$ \\
$\frac{\partial}{\partial \mathbf{x}}(\mathbf{x}^\top \mathbf{x}) = 2\mathbf{x}$ \\
$\frac{\partial}{\partial \mathbf{x}}(\mathbf{x}^\top \mathbf{A}\mathbf{x}) = (\mathbf{A}^\top + \mathbf{A})\mathbf{x}$ \quad
$\frac{\partial}{\partial \mathbf{x}}(\mathbf{b}^\top \mathbf{A}\mathbf{x}) = \mathbf{A}^\top \mathbf{b}$ \quad
$\frac{\partial}{\partial \mathbf{X}}(\mathbf{c}^\top \mathbf{X} \mathbf{b}) = \mathbf{c}\mathbf{b}^\top$ \quad
$\frac{\partial}{\partial \mathbf{x}}(\| \mathbf{x}-\mathbf{b} \|_2) = \frac{\mathbf{x}-\mathbf{b}}{\|\mathbf{x}-\mathbf{b}\|_2}$ \\
$\frac{\partial}{\partial \mathbf{x}}(\|\mathbf{x}\|^2_2) = \frac{\partial}{\partial \mathbf{x}} (\mathbf{x}^\top \mathbf{x}) = 2\mathbf{x}$ \quad
$\frac{\partial}{\partial \mathbf{X}}(\|\mathbf{X}\|_F^2) = 2\mathbf{X}$  \quad \quad
$\frac{\partial}{\partial \mathbf{x}}||\mathbf{x}||_1 = \frac{\mathbf{x}}{|\mathbf{x}|}$ \\
$\frac{\partial}{\partial \mathbf{x}}(\|\mathbf{Ax - b}\|_2^2) = \mathbf{2(A^\top Ax-A^\top b)}$ \quad
$\frac{\partial}{\partial \mathbf{X}}(|\mathbf{X}|) = |\mathbf{X}|\cdot \mathbf{X}^{-1}$ \\
$\frac{\partial}{\partial x}(\mathbf{Y}^{-1}) = -\mathbf{Y}^{-1} \frac{\partial\mathbf{Y}}{\partial x} \mathbf{Y}^{-1}$
\section*{Parametric Density Estimation}
\subsection*{Maximum Likelihood (MLE)}
Likelihood: $\mathbb{P}[\mathcal{X}|\theta]=\prod_{i\leq n}p(x_i|\theta)$\\
Find: $\hat{\theta}\in \argmax_\theta \mathbb{P}[\mathcal{X}|\theta]$\\
Procedure: solve $\nabla_\theta \log \mathbb{P}[\mathcal{X}|\theta]\equiv 0$\\
Consistent: converges to best $\theta_0$.

\subsection*{Maximum A Posteriori (MAP)}
Assume prior $\mathbb{P}(\theta)$\\
Find: $\hat{\theta}\in \argmax_\theta P(\theta|\mathcal{X})$\\
$\quad \quad \quad =\argmax_\theta P(\mathcal{X}|\theta)P(\theta)$\\
Solve $\nabla_\theta log P(\mathcal{X}|\theta)P(\theta)=0$\\
Note:  $ P(\mathcal{Y}|\mathcal{X}, \beta) \sim \mathcal{N}(X^T\beta, \sigma^2) \rightarrow \argmax_\theta log(P(\mathcal{Y}|\mathcal{X}, \beta)) =  \argmax_\theta -\frac{1}{2\sigma^2}||Y-X^T\beta||^2$
\subsection*{Bayesian learning}
$p(X=x|data) = \int p(x, \theta | data) d\theta = \int p(x|\theta)p(\theta|data)d\theta$ \\
Estimate gaussian: $X \sim \mathcal{N}(\mu, \sigma^2),$ \\
$P(\mu) \sim \mathcal{N}(\mu_0, \sigma_0^2)$ then $\sigma_n^2 = \frac{\sigma^2 \sigma_0^2}{n\sigma_0^2 + \sigma^2},$ $ \mu_n = \frac{n\sigma_0^2}{n\sigma_0^2 + \sigma^2} \hat \mu_n + \frac{\sigma^2}{n\sigma_0^2 + \sigma^2}\mu_0$ with \\
$\hat \mu_n = \frac{1}{n}\sum_{i=1}^n x_i$
\subsection*{Bayesian density learning}
Prior Knowledge of $p(\theta)$,\\
Find Posterior Density: $p(\theta|\mathcal{X})$.\\
$\mathcal{X}^n=\{x_1, \cdots, x_n\}$\\
$p(\theta|\mathcal{X}^n)=\frac{p(x_n|\theta)p(\theta|\mathcal{X}^{n-1})}{\int p(x_n|\theta)p(\theta|\mathcal{X}^{n-1}) d\theta}$
% Difficult \& needs prior knowledge. But better against overfitting.


\subsection*{Frequentist vs Bayesian}
Bayes (MAP): allows priors, provides distribution when estimating parameters, requires efficient integration methods when computing posteriors, prior often induces regularization term \\
Frequentist method (MLE): does not allow priors, provides a single point when estimating parameters, requires only differentiation methods, MLE estimators are consistent, equivariant, asymptotically normal, asymptotically efficient. 

\section*{Optimization}
\subsection*{Gradient Descent}
$\theta^{\mathrm{new}}\leftarrow\theta^{\mathrm{old}}-\eta\nabla_{\theta}\mathcal{L}$\\
Conv. isn't guaranteed.
Less zigzag by adding momentum: $\theta^{(l+1)}\leftarrow\theta^{(l)}-\eta\nabla_{\theta}\mathcal{L}+\mu(\theta^{l}-\theta^{(l-1)})$

\subsection*{Newton's Method (opt. grad. descent)}
Use 2nd order derivation. (Hessian)
$\theta^{\mathrm{new}}\leftarrow\theta^{\mathrm{old}}-\eta(\nabla_{\theta}\mathcal{L}/\nabla^2_{\theta}\mathcal{L})$\\
$H=\nabla^2_{\theta}\mathcal{L}$ has to be p.d (convex func).
Find this by setting derivative wrt. t of taylor expansion of loss function to 0. \\
Taylor:\\
$f(x+t)=f(x)+t f'(x)+\frac{1}{2}f''(x)t^2$

\section*{Data Types}
monadic: $X: O \rightarrow \mathbb{R}^d$ \\
dyadic: $X: O^{(1)} x O^{(2)} \rightarrow \mathbb{R}^d $.\\
pairwise: $X: O^{(1)} x O^{(1)} \rightarrow \mathbb{R}^d$ \\
polyadic data: $X: O^{(1)} x O^{(2)} x O^{(3)} \rightarrow \mathbb{R}^d $ \\
nominal = qualitative (sweet, sour ...), ordinal = absolute order of items, quantitative = numbers
\section*{Risks and Losses}
Expected Risk:\\ $R(\hat c) = \sum_{y \leq  k}P(y)\mathbb{E}_{P(x|y)}[\mathbb{I}_{{\hat c(x) \neq y}}| Y = y]$ or 
$R(\hat c) = P(\hat c (X) \neq c(X))$, c opt. class.

Empirical Risk Minimizer (ERM) $\hat{f}$:\\
$\hat{f} \in \argmin_{f \in \mathcal{C}} \hat{R}(\hat{f}, Z^{train})$\\
$\hat{R}(\hat{f}, Z^{train}) = \frac{1}{n} \sum_{i=1}^n Q(Y_i, \hat{f}(X_i))$\\
$\hat{R}(\hat{f}, Z^{test}) = \frac{1}{m} \sum_{i=n+1}^{n+m} Q(Y_i, \hat{f}(X_i))$